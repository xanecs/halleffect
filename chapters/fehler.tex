\chapter{Fehlerbetrachtung}
Insgesamt sind die Messwerte gut. Sie bestätigen den entsprechenden Zusammenhang von Querstrom und Hallspannung bei konstantem B-Feld.
Jedoch bereitete vor allem der Versuch mit dem Silberplättchen einige Schwierigkeiten bei der Genauigkeit der Messung.

\section{Silberplättchen}
Zunächst war es schwierig, ein Messgerät für die sehr hohe Stromstärke des Querstroms zu finden. Alle vorhandenen „normalen“ Amperemeter sind bis max. \SI{10}{\ampere} ausgelegt. Das Labornetzteil hat zwar eine eingebaute Anzeige für die Stromstärke, diese sind allerdings erfahrungsgemäß nicht sehr genau. Schließlich haben wir die Stromstärke mit der Stromzange gemessen. Sie misst das Drehmagnetfeld um einen stromdurchflossenen Leiter. Es kann daher sein, dass durch den Elektromagneten Messfehler entstanden sind.

Auch die Messung der sehr kleinen Hallspannung war problematisch. Die Mikrovoltbox hat nur eine Auflösung von \SI{1}{\micro\volt}. Da die Spannung selbst bei \SI{20}{\ampere} Querstrom nur \SI{6}{\micro\volt} beträgt, sind die Fehler sehr hoch.

Die ermittelte Hallkonstante liegt dennoch sehr nahe am in der Anleitung angegebenen Wert.

\section{Germaniumplättchen}
Die Messung mit dem Germaniumplättchen ist wesentlich genauer möglich, da zum einen die geringer Querstromstärke mit einem Multimeter genau gemessen werden kann, und zum anderen die Hallspannung im \SI{}{\milli\volt}-Bereich liegt. Sie kann also ebenfalls genauer bestimmt werden. Dies führt zu einem sehr kleinen Fehler bei der ermittelten Hallkonstante.

Die daraus berechnete Ladungsträgerkonzentration liegt allerdings nicht im Bereich, der in der Anleitung angegeben ist. Dort werden für $n$ \SIrange{6e20}{9e20}{\per\cubic\meter} angegeben. Unser Wert beträgt $n = \SI{4.80 \pm 0.08e20}{\per\cubic\meter}$. Da der Wert für die Hallkonstante allerdings sehr gut innerhalb des in der Anleitung angegebenen Bereichs liegt und für die Berechnung der Ladungsträgerkonzentration außer $R_H$ nur Konstanten und Angaben aus der Anleitung verwendet wurden, gehen wir hier nicht von einem Fehler aus.